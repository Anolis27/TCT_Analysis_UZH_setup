\chapter{Conclusion}
\label{conclusion}

The work presented in this thesis aimed to explore two techniques by building an experimental setup designed to detect defects in silicon strip detectors using a camera sensitive in the SWIR range. The first technique consisted in illuminating the sensors with infrared LEDs, while the second relied on forward biasing the sensors to observe their electroluminescence. The success of these approaches could provide a useful inspection tool for the assembly line of 2S modules at IIHE, within the scope of the phase-2 upgrade project of the CMS silicon tracker.

First, the experimental setup was designed and implemented inside a light-isolated cabinet. This required selecting appropriate optical components, designing and building a pneumatic system to dry the air in the cabinet, as well as creating and 3D-printing mechanical parts. The firmware of an electronic board for control and monitoring of the setup was also programmed, although its full potential was not fully exploited.

A validation of the setup was then carried out. It confirmed that the system was capable of efficiently detecting defects in silicon solar cells, as well as in a wafer fragment taken from the same wafer as a CMS phase-0 sensor. However, it was concluded that transmission-based transparency analysis would not be applicable to assembled 2S modules, due to the superposition of two aluminium backplanes.

Subsequently, reflection-based transparency analysis was performed on both phase-0 and phase-2 sensors. Unfortunately, these experiments did not reveal significant defects. This study was limited by material constraints of the setup — such as the need to bring the sensors very close to the camera — as well as by the unknown nature of the expected defects. Nevertheless, this method could still be of interest for detecting potential cracks, but this remains to be demonstrated for this type of sensor.

Finally, phase-0 and phase-2 sensors of CMS were forward-biased, producing an EL signal along the bias and guard rings. However, currents much higher than the nominal operating current of 2S modules were required, which made it necessary to bypass the module electronics. Furthermore, the inhomogeneity of the EL observed in these sensors does not provide conditions as favourable for defect detection as in the case of solar cells. Overall, these limitations make systematic inspection of modules using this technique impractical in its current form for the assembly line.

In summary, these analyses clarified the potential and limitations of transparency imaging and EL for detecting defects in silicon strip sensors. Building on this work, synchronizing the camera with HV pulses could potentially yield high-quality EL images with minimal electrical stress on module electronics, while photoluminescence imaging could offer a complementary, contact-free approach already used in photovoltaic inspection.