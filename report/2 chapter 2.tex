\chapter{Analysis program for TCT measurments}
\label{chap2}

This section describes the development of a Python program for the data analysis of the TCT setup at UZH, starting from an initial script written by ................. The current state of the project can be found on the \href{https://github.com/Anolis27/TCT_Analysis_UZH_setup}{following GitHub repository}. The program was developed and optimized to analyse TCT measurements on Ti-LGADs, but a version of the initial code has also been adapted to analyse TCT data on CASSIA sensors. Furthermore, some parts of the code can be reused for the analysis of TCT measurements on other types of sensors.

\begin{figure}[h!]
  \centering
  \resizebox{0.9\textwidth}{!}{%
    \begin{forest}
      for tree={
        font=\ttfamily,
        grow'=0,
        child anchor=west,
        parent anchor=south,
        anchor=west,
        calign=first,
        edge path={
          \noexpand\path [draw, \forestoption{edge}] (!u.parent anchor) |- (.child anchor)\forestoption{edge label};
        },
        before typesetting nodes={
          if n=1
            {insert before={[,phantom]}}
            {}
        },
      }
      [TCT$\_$Analysis$\_$UZH$\_$setup/
        [.gitignore]
        [README.md]
        [CASSIA$\_$ANALYSIS/
          [Analysis$\_$Cassia$\_$1ch.py]
          [README$\_$CASSIA.txt]
        ]
        [LGAD$\_$ANALYSIS/
          [Analysis$\_$2.tex]
          [amplitude.py]
          [charge$\_$collection.py]
          [config.py]
          [data$\_$manager.py]
          [interpad.py]
          [main.py]
          [plot$\_$2D$\_$maps.py]
          [plot$\_$saved$\_$results.py]
          [timing.py]
        ]
      ]
    \end{forest}
  }
  \caption{Structure of the GitHub repository for the TCT analysis program developed at UZH.}
  \label{fig:repo_structure}
\end{figure}

\section{Structure of the GitHub repository}

The structure of the GitHub repository is shown in the diagram bshown on Figure \ref{fig:repo_structure}. The repository contains two main folders: \texttt{CASSIA\_ANALYSIS} and \texttt{LGAD\_ANALYSIS}, which contain the analysis scripts for CASSIA sensors and Ti-LGADs, respectively. The main contribution to this project is the development of the \texttt{LGAD\_ANALYSIS} folder, which includes various Python scripts for different aspects of the TCT data analysis. 


\section{Description of the Ti-LGAD analysis program}

This section provides a detailed description of the usage of the Ti-LGAD analysis program. The main script of the program is \texttt{main.py}, which serves as the entry point for the analysis. The program is designed to be modular, with different scripts handling specific tasks such as amplitude analysis, charge collection analysis, timing analysis, and plotting of results. The different parameters for the analysis can be configured in the \texttt{config.py} file. The other scripts contain the functions, often heavily centred on the Ti-LGAD measurment, needed to perform the analysis. 

The software of the setup at UZH outputs a folder with multiple subfolders, containing the data files needed for the analysis, but only a few of these files are needed for the analysis. The \texttt{positions.pickle}, \texttt{measured$\_$data.sqlite} and \texttt{parsed$\_$from$\_$waveforms.sqlite} files has to be added in a folder following the same structure as \newline \texttt{LGAD$\_$ANALYSIS/Data/SensorName/BiasVoltage/}. % (e.g. \texttt{LGAD$\_$ANALYSIS/Data/W5$\_$V1$\_$TW5/100V/}).

\begin{figure}[h!]
    \centering
    \begin{subfigure}[t]{0.45\linewidth}
        \centering
        \includegraphics[width=\linewidth]{images/chap2/TI_LGAD_sample.png}
        \caption{Microscope image of a Ti-LGAD sample with the opening in the metalisation for the TCT measurement.}
        \label{fig:TI_LGAD_sample}
    \end{subfigure}
    \hfill
    \begin{subfigure}[t]{0.5\linewidth}
        \centering
        \includegraphics[width=\linewidth]{images/chap2/2d_collected_charge.png}
        \caption{2D plot of the collected charge in the area of the opening in the metalisation.}
        \label{fig:2d_collected_charge}
    \end{subfigure}
    \caption{Wirebonding between the two sensors of a phase-0 module. Each pair of strips corresponds to a number printed on the edge ring.}
    \label{fig:data_acquisition}
\end{figure}

The program has been developped for area scans around the opening of the metalisation of the Ti-LGADs samples, as shown on Figure \ref{fig:data_acquisition}. The typical config for the data acquisition is:
\begin{itemize}
  \item step = 5e-6
  \item X SPAN = 100e-6
  \item Y SPAN = 300e-6
  \item LASER DAC = 500
  \item N TRIGGERS = 100
  \item Save Waveforms = False
\end{itemize}

It is important to notice that the x and y axis are usually inverted in the code. This was originating from the initial script written by ......... and kept for the rest of the developpement of the code. 

\subsection{\texttt{main.py}}

The \texttt{main.py} file contains the main functions used to run the analysis. Depending on the analysis the user wants to perform, the required functions must be uncommented. The different functions are grouped according to the type of analysis: amplitude analysis, charge collection analysis, timing analysis, or interpad region analysis. Two additional sections are also dedicated to debugging and the configuration of the parameterers for the noise filter.

\subsection{\texttt{config.py}}

This python file contains the main parameters to run the analysis. It is composed of differents classes: Paths, Filters, InterpadConfig, Subplots, PlotsConfig, and Colors. The following subsections will give a brief word on these classes. 

\subsubsection{\texttt{Paths}}
This class is where you indicate the paths of the data the user want to analyse or the paths of the output folders. 
\texttt{DATA$\_$ROOT} is the path of the \texttt{Data} folder and need to be adapted depending where the user is doiong the analysis on his computer. 
\texttt{PREFERRED$\_$BASE$\_$DIR} is the paths of the data of a precise measurement the user wants to analyse. This is required for all the functions requireing \texttt{DATAFILE} (the \texttt{parsed$\_$from$\_$waveforms.sqlite}) and/or \texttt{POSITIONS} (the \texttt{positions.pickle} file) as argument.
The rest of thius class defines a method to find automaticaly the \texttt{BASE$\_$DIR} if the \texttt{PREFERRED$\_$BASE$\_$DIR} does not reffers to an existing folder. This method is used at the beginning of the \texttt{main()} function.

\subsubsection{\texttt{Filters}}
This class is used to define the different thresholds to cut off the noise and select the actual data. These values can be chosen by visualising them with the functions \texttt{plot$\_$amplitude()}, \texttt{plot$\_$amplitude$\_$against$\_$t$\_$peak()} and \texttt{plot$\_$time$\_$difference$\_$t50()}. 

\subsubsection{\texttt{InterpadConfig}}
This class defines the different parameters used for the analysis in the interpad region. The x interval on the 2D plot like the one given in Figure \ref{fig:2d_collected_charge} defining the interpad region is given by the parameters \texttt{INTERPAD$\_$MIN} and \texttt{INTERPAD$\_$MAX}. The stepsize used during the measurement must also be given here. The y position where it is considered as the middle of the opening has also to be given. The \texttt{INTERPAD$\_$FRACTION} defines at which fraction of the sigmoid fit in \texttt{get$\_$interpad$\_$distance()}. \texttt{INTERPAD$\_$TIMING$\_$SCALE} defines the limits of the y axis for the plots of \texttt{plot$\_$time$\_$resolution$\_$interpad$\_$region$\_$everything()} (or the ones from the saved results).

\subsubsection{\texttt{Subplots}}
Boolean values if the subplots for the \texttt{\dots plot$\_$everything()} functions have to be saved in the pdf file. This has currently not been implemented for all such sublots but can easily be done.

\subsubsection{\texttt{PlotsConfig}}
Configures the max values for the 2D heatmaps scale or the time interval in which the \texttt{plot$\_$time$\_$difference$\_$t50()} has to focus.

\subsubsection{\texttt{Colors}}
Defines the list of colors used for the plots.










\subsection{\texttt{amplitude.py}}
This python file defines the functions used for the amplidue analysis. A brief description of each function is given in the following subsections.

\subsubsection{\texttt{plot$\_$amplitude()}}
Plots the histogram of the amplitudes recorded during the datataking, after filtering the noise on the time difference between the two pulses of the same trigger and the peaktime (see Sections \ref{} and \ref{sec:peak_t}). This enables to visualise the \texttt{AMPLITUDE$\_$THRESHOLD} to cut the noise and an example of the output plot is given in Figure \ref{fig:plot_amplitude}.

\begin{figure}[h]
  \centering
  \includegraphics[width=0.9\linewidth]{images/chap2/Amplitude_distribution.png}
  \caption{Amplitude histogram of the data after filtering on the time difference and the peaktime.}
  \label{fig:plot_amplitude}
\end{figure}

\subsubsection{\texttt{plot$\_$noise()}}
Plots the histogram of the noise recorded during the measurement. Figure \ref{fig:plot_noise} give an example of such a plot.

\begin{figure}[h]
  \centering
  \includegraphics[width=0.6\linewidth]{images/chap2/Noise_distribution.png}
  \caption{Noise histogram.}
  \label{fig:plot_noise}
\end{figure}

\subsubsection{\texttt{plot$\_$amplitude$\_$against$\_$t$\_$peak()}}
\label{sec:peak_t}
Plots the amplitude against the peak time calculated by default with $ t_{peak} = t_{90} + 0.5 \cdot t_{over 90}$. The time variable (here $t_{90}$ and $t_{over 90}$) can be changed by gining it as argument to the function. This function can be used to define the \texttt{Filters} parameters based on the peak time. An example of such a plot is given in Figure \ref{fig:plot_peak_time}.

\begin{figure}[h]
  \centering
  \includegraphics[width=0.9\linewidth]{images/chap2/Amplitude_vs_peak_time.png}
  \caption{Amplitude against the peak time.}
  \label{fig:plot_peak_time}
\end{figure}

\subsubsection{\texttt{plot$\_$amplitude$\_$of$\_$one$\_$pad()}}
Plots a histogram of the amplitudes recorded in one channel, fits a gaussian curve and returns the mean amplitude and the standard deviation recorded in the channel. Figure \ref{fig:plot_ampl_1_pad} gives an exemple of such a plot (the values in the legend are rounded).

\begin{figure}[h]
  \centering
  \includegraphics[width=0.6\linewidth]{images/chap2/Amplitude_one_pad.png}
  \caption{Amplitude histogram for one channel.}
  \label{fig:plot_ampl_1_pad}
\end{figure}

\subsubsection{\texttt{plot$\_$amplitude$\_$everything()}}
Generates a pdf file with the plots of \texttt{plot$\_$amplitude$\_$of$\_$one$\_$pad()} for each channel, bias voltage and sensor located in \texttt{Data/} folder. The last plot saved in the pdf represents the mean amplitude against the bias voltage for each channel. Figure \ref{fig:plot_ampl_everything} is an example of this plot for TCT measurements on different Ti-LGAD samples. The results are of this plot are saved in \texttt{.pkl} files in the save directory, defined in the \texttt{config.py}, in order to save run time in case analysis has to be refined for one sensor. This is explained in Section \ref{sec:save_results}.

\begin{figure}[h]
  \centering
  \includegraphics[width=0.6\linewidth]{images/chap2/Amplitude_everything.png}
  \caption{Mean amplitude against bias voltage for each channel.}
  \label{fig:plot_ampl_everything}
\end{figure}

\subsubsection{\texttt{plot$\_$2D$\_$separate()}}
This function outputs the plots on Figure \ref{fig:plot_2d_sep}. The right hand side of the Figure represents the 2D plot of the amplitude recorded in each channel and the left hand side represents a projection of this amplitude on the y axis. This is a not modified function of the original code and must be adapted to correct the plots on the left hand side. In fact, this projection is calculated with avrages on the whole data in each y position, taking the noise and the 0 amplitude data into account, resulting in a underestimation of the real amplitude abserved in each channel and un large standard deviation (resulting in large error bars).

\begin{figure}[h]
  \centering
  \includegraphics[width=0.5\linewidth]{images/chap2/plot_2d_separate.png}
  \caption{Output of \texttt{plot$\_$2D$\_$separate()}.}
  \label{fig:plot_2d_sep}
\end{figure}

\subsubsection{\texttt{project$\_$onto$\_$y$\_$one$\_$channel$\_$amplitude()}}
Returns a dictionnary with the amplitude projection of in the interpad region defined in the \texttt{config.py}. This function was created to compute the interpad distance with the amplitude (see Section \ref{sec:interpad}). The fact that the name of the function refers to a projection along the y axis and actually project along the x axis on Figure \ref{fig:2d_collected_charge} is because there is an inversion between the axis compared to the actual measured data and can lead to some confusions.

\subsubsection{\texttt{plot$\_$amplitude$\_$along$\_$y$\_$axis()}}

This function plots the projection of the amplitude and the collected charge in each channel for one selected y value in the middle of the opening (defined in \texttt{config.py}) along the y axis of Figures \ref{fig:2d_ampl} and \ref{fig:2d_collected_charge}. Figure \ref{fig:plot_projection} represents an output of this function.

\begin{figure}[h]
  \centering
  \includegraphics[width=0.9\linewidth]{images/chap2/project_along_y.png}
  \caption{Projection of the normalised amplitude and the collected charge on x for a given y of the 2D plots on Figures \ref{fig:2d_ampl} adn \ref{fig:2d_collected_charge}.}
  \label{fig:plot_projection}
\end{figure}

\subsection{\texttt{interpad.py}}
\label{sec:interpad}

\subsection{\texttt{data$\_$manager.py}}
\subsubsection{\texttt{save$\_$results()}}
\label{sec:save_results}

% WORD IF HAVE DO MEASURE WITH VERTIVAL LGAD OR OTHER DEVICE